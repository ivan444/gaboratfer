\documentclass{article}
\usepackage[utf8]{inputenc}
\usepackage[croatian]{babel}
\usepackage[T1]{fontenc}
\usepackage{lmodern}
\usepackage{algorithmic}
\usepackage{algorithm}
\usepackage{longtable}
\usepackage{graphicx}
\usepackage{booktabs}
% Da bi se promjenio stil citiranja umjesto:
% [authoryear, round]
% staviti:
% [numbers, square]
\usepackage[authoryear, round]{natbib}
\usepackage{amsmath}
\usepackage{subfig}
\usepackage{fixltx2e}

\begin{document}
\title{Izlučivanje značajki gaborovim filterom}
\author{Tomislav Reicher \and Krešimir Antolić \and Igor Belša \and Marko Ivanković \and Ivan Krišto \and Maja Legac \and Tomislav Novak}

\maketitle

\tableofcontents

\section{Uvod}
Face representation based on Gabor features have attracted much attention and
achieved great success in face recognition area for the advantages of the Gabor
filters. However, Gabor features currently adopted by most systems are redundant
and too high dimensional. In this paper, we propose a face recognition method
using AdaBoosted Gabor features, which are not only low dimensional but also
discriminant. The main contribution of the paper lies in two points: (1) AdaBoost
is successfully applied to face recognition by introducing the intra-face and
extra-face difference space in the Gabor feature space; (2) An appropriate
re-sampling scheme is adopted to deal with the imbalance between the amount of
the positive samples and that of the negative samples. By using the proposed
method, only hundreds of Gabor features are selected. Experiments on FERET
database has shown that these hundreds of Gabor features are enough to achieve
good performance comparable to that of methods using the complete set of Gabor
features.

Tekst je izvučen iz \citep{Yang04facerecognition}.

Ovo je na brzinu složeno\ldots uskoro će biti lijepo ;D.

\section{O literaturi}
Literaturu izvlačite sa citeseerxa ili google schoolara jer vam oni odmah daju
i bibtex članka koji samo kopirate u file \emph{literatura.bib}.

Primjer citiranja: \verb|\citep{Yang04facerecognition}|.

\bibliography{literatura}
\bibliographystyle{plainnat}


\end{document}
